% This file is part of the multi-tex ECE 486 Final Project Report 
% file name: report.tex

% YOU CAN COMPILE FROM THIS FILE, THIS IS THE MAIN FILE, read the following

% included files:
% (Makefile)
% Makefile -- run make Makefile (on Mac or Linux), it takes care of LaTeX compilation

% (*.PDF)
% report.pdf -- (this file) report file, print it out and submit it

% (*.TEX)
% report.tex -- main file, pdflatex this file (tested on Mac and Linux)
% 0-title-page.tex -- title page
% 1-introduction.tex -- chapter 1 
% 2-mathematical-model.tex -- chapter 1 (lagrange equations of motion) and chapter 4 (linearisation)
% 3-full-state-feedback-control-friction-compensation.tex -- chapter 2 and chapter 4 (two state and three state feedback controller design)
% 4-full-state-feedback-control-decoupled-observer.tex -- chapter 4 (controller design) and chapter 5 (observer design for estimated state)
% 5-conclusions.tex -- conclusion
% 6-extra-credit.tex -- (optional) add thsese pages if you have demoed chapter 6 and chapter 7

\documentclass[11pt,letterpaper,twoside]{article}
\usepackage{mathptmx} %Times font
\usepackage{graphicx}

% packages published by the American Mathematical Society
\usepackage{mathtools} % mathtools loads amsmath and adds some useful tools. See, http://tex.stackexchange.com/questions/43860/mathtools-vs-amsmath
\usepackage{amssymb}

% for the title page
\newcommand{\HRule}{\rule{\linewidth}{0.5mm}}

% my favorite font: charter
%\usepackage{charter}

% create hyper-links within the document
\usepackage{hyperref}

% for colored links
\usepackage{xcolor}
\definecolor{illinois_professional_blue}{cmyk}{60, 40, 5, 0}
\definecolor{illinois_professional_orange}{cmyk}{3, 55, 100, 0}
\definecolor{illinois_bold_blue}{cmyk}{100, 86, 24, 9}
\definecolor{illinois_bold_orange}{cmyk}{0, 62, 99, 0}

\hypersetup{colorlinks=true, linktoc=all, linkcolor=illinois_professional_blue,}

% adds bibliography to table of contents
% \usepackage[nottoc]{tocbibind}

% segregating bibliography and references
% \usepackage{multibib}
% \newcites{ltex}{References}

% every section starts on a new page
% redefining section
\let\stdsection\section\renewcommand\section{\newpage\stdsection} % Leslie Lamport warns it is better to create newcommand rather than overwrite existing command, especially LaTeX built-ins

\begin{document}
% line 1: no numbering for table of contents
\addtocontents{toc}{\protect\thispagestyle{empty}}

% This file is part of the multi-tex ECE 486 Final Project Report 
% file name: 0-title-page.tex

% YOU DO NOT COMPILE FROM THIS SINGLE FILE, read the following

% included files:
% (Makefile)
% Makefile -- run make Makefile (on Mac or Linux), it takes care of LaTeX compilation

% (*.PDF)
% report.pdf -- report file, print it out and submit it

% (*.TEX)
% report.tex -- main file, pdflatex this file (tested on Mac and Linux)
% 0-title-page.tex -- (this file) title page
% 1-introduction.tex -- chapter 1 
% 2-mathematical-model.tex -- chapter 1 (lagrange equations of motion) and chapter 4 (linearisation)
% 3-full-state-feedback-control-friction-compensation.tex -- chapter 2 and chapter 4 (two state and three state feedback controller design)
% 4-full-state-feedback-control-decoupled-observer.tex -- chapter 4 (controller design) and chapter 5 (observer design for estimated state)
% 5-conclusions.tex -- conclusion
% 6-extra-credit.tex -- (optional) add thsese pages if you have demoed chapter 6 and chapter 7
 
\begin{titlepage}
	\begin{center}

		\textsc{\LARGE University of Illinois at Urbana-Champaign}\\[1.5cm]

		% Title
		\HRule\\[0.4cm]
		{\huge \bfseries ECE 486 : Final Project Report \\[0.4cm] }
		\uppercase{Fall 2014}\\[0.5cm]

		\HRule\\[1.5cm]
                
                % Add your info and your lab partner's info. Also include your section and TA's name
		\noindent
		\begin{minipage}{0.4\textwidth}
			\begin{flushleft} \large
				\textbf{Daniel McKeogh} % can we remove the names of students? just insert a name like FirstName1 LastName1 for lab partner?
			\end{flushleft}
		\end{minipage}%
		\begin{minipage}{0.4\textwidth}
			\begin{flushright} \large
				\textbf{Rohan R. Arora} % FirstName2 LastName2
			\end{flushright}
		\end{minipage}
		\\~\\
		\textit{Teaching Assistant}: Y\"{u}n Han\\ % change it to TA's Name?
		\textit{Day of Laboratory Section}: Tuesday % also this line depends

		\vfill

		% Bottom of the page
		{\large \today} % you can explicitly use a date if you don't like \today

	\end{center}
\end{titlepage}










\tableofcontents

% line 2: no numbering for table of contents
\setcounter{page}{0}
\section{Introduction}

\subsection{Sensors}
\subsection{Actuators}
\subsection{Equilibrium Positions}
\subsection{Implementation}

% This file is part of the multi-tex ECE 486 Final Project Report 
% file name: 2-mathematical-model.tex

% YOU DO NOT COMPILE FROM THIS SINGLE FILE, read the following

% included files:
% (Makefile)
% Makefile -- run make Makefile (on Mac or Linux), it takes care of LaTeX compilation

% (*.PDF)
% report.pdf -- report file, print it out and submit it

% (*.TEX)
% report.tex -- main file, pdflatex this file (tested on Mac and Linux)
% 0-title-page.tex -- title page 
% 1-introduction.tex -- chapter 1 
% 2-mathematical-model.tex -- (this file) chapter 1 (lagrange equations of motion) and chapter 4 (linearisation)
% 3-full-state-feedback-control-friction-compensation.tex -- chapter 2 and chapter 4 (two state and three state feedback controller design)
% 4-full-state-feedback-control-decoupled-observer.tex -- chapter 4 (controller design) and chapter 5 (observer design for estimated state)
% 5-conclusions.tex -- conclusion
% 6-extra-credit.tex -- (optional) add thsese pages if you have demoed chapter 6 and chapter 7

\section{Mathematical Model}

\subsection{Derivation of differential equations from Lagrangian}
\subsection{Linearization into State Space Form}

\section{Full State Feedback Control with Friction Compensation}

\subsection{Development of the PD Control with Friction Compensation}
\subsection{Mathematical Proof}
\subsection{Robustness Comparisons}
\subsection{System Behavior}

% This file is part of the multi-tex ECE 486 Final Project Report 
% file name: 4-full-state-feedback-control-decoupled-observer.tex

% YOU DO NOT COMPILE FROM THIS SINGLE FILE, read the following

% included files:
% (Makefile)
% Makefile -- run make Makefile (on Mac or Linux), it takes care of LaTeX compilation

% (*.PDF)
% report.pdf -- report file, print it out and submit it

% (*.TEX)
% report.tex -- main file, pdflatex this file (tested on Mac and Linux)
% 0-title-page.tex -- title page 
% 1-introduction.tex -- chapter 1 
% 2-mathematical-model.tex -- chapter 1 (lagrange equations of motion) and chapter 4 (linearisation)
% 3-full-state-feedback-control-friction-compensation.tex -- chapter 2 and chapter 4 (two state and three state feedback controller design) 
% 4-full-state-feedback-control-decoupled-observer.tex -- (this file) chapter 4 (controller design) and chapter 5 (observer design for estimated state)
% 5-conclusions.tex -- conclusion
% 6-extra-credit.tex -- (optional) add thsese pages if you have demoed chapter 6 and chapter 7

\section{Full State Feedback Control with Decoupled Observer}

\subsection{Theoretical Background}
\subsection{Derivation}
\subsection{Robustness}
\subsection{System Behavior}

\section{Conclusions}

% This file is part of the multi-tex ECE 486 Final Project Report 
% file name: 6-extra-credit.tex

% YOU DO NOT COMPILE FROM THIS SINGLE FILE, read the following

% included files:
% (Makefile)
% Makefile -- run make Makefile (on Mac or Linux), it takes care of LaTeX compilation

% (*.PDF)
% report.pdf -- report file, print it out and submit it

% (*.TEX)
% report.tex -- main file, pdflatex this file (tested on Mac and Linux)
% 0-title-page.tex -- title page 
% 1-introduction.tex -- chapter 1 
% 2-mathematical-model.tex -- chapter 1 (lagrange equations of motion) and chapter 4 (linearisation)
% 3-full-state-feedback-control-friction-compensation.tex -- chapter 2 and chapter 4 (two state and three state feedback controller design) 
% 4-full-state-feedback-control-decoupled-observer.tex -- chapter 4 (controller design) and chapter 5 (observer design for estimated state)
% 5-conclusions.tex -- conclusion
% 6-extra-credit.tex -- (this file) (optional) add thsese pages if you have demoed chapter 6 and chapter 7

\section{Extra Credit: Up and Down Stabilizing Control}

\section{Extra Credit: Swing-Up Control}


% references
%\bibliographystyleltex{unsrt}
%\bibliographyltex{bib}

%\renewcommand{\refname}{Bibliography}

% bibliography
%\bibliographystyle{plain}
%\bibliography{bib}

\end{document}





%%% Local Variables:
%%% mode: latex
%%% TeX-master: t
%%% End:
